\documentclass[11pt,a4paper]{article}
\usepackage[utf8]{inputenc}
\usepackage[francais]{babel}
\usepackage[T1]{fontenc}
\usepackage{amsmath}
\usepackage{amsfonts}
\usepackage{amssymb}
\usepackage{graphicx}
\usepackage{eurosym}
\usepackage{lmodern}
\usepackage{listings}
\title{Rapport projet F1}
\author{Justine Doutreloux, Simon Fauconnier, Steve Henriquet, Lucie Hermand}
\usepackage{hyperref}
\hypersetup{
    colorlinks,
    citecolor=black,
    filecolor=black,
    linkcolor=black,
    urlcolor=black
}
\lstset{ 
  basicstyle=\footnotesize,        % the size of the fonts that are used for the code
  breakatwhitespace=false,         % sets if automatic breaks should only happen at whitespace
  breaklines=true,                 % sets automatic line breaking
  captionpos=b,                    % sets the caption-position to bottom
  extendedchars=true,              % lets you use non-ASCII characters; for 8-bits encodings only, does not work with UTF-8
  %frame=single,	                   % adds a frame around the code
  keepspaces=true,                 % keeps spaces in text, useful for keeping indentation of code (possibly needs columns=flexible)
  language=c,                 % the language of the code
  numbers=left,                    % where to put the line-numbers; possible values are (none, left, right)
  numbersep=5pt,                   % how far the line-numbers are from the code
  showspaces=false,                % show spaces everywhere adding particular underscores; it overrides 'showstringspaces'
  showstringspaces=false,          % underline spaces within strings only
  showtabs=false,                  % show tabs within strings adding particular underscores
  stepnumber=1,                    % the step between two line-numbers. If it's 1, each line will be numbered
  tabsize=2,	                   % sets default tabsize to 2 spaces
}
\date{Année scolaire 2017-2018}
\begin{document}
\maketitle

\newpage
\tableofcontents

\newpage
\part{Introduction}

%Création d'un programme en C système sous CentOs7. \newline 
Ce programme simule un grand prix de formule 1 comprenant les séances d'essais, les qualifications et la course en elle-même. Vingt voitures se disputent la victoire sur un circuit divisé en trois secteurs et dont la taille, décidée par l'utilisateur, est passée en paramètre du programme.
Pour ce programme, il nous a été demandé d'utiliser la mémoire partagée comme moyen de communication inter-processus ainsi que les sémaphores pour synchroniser l'accès à la mémoire partagée.\par
Les séances d'essais se divisent en trois périodes, les deux premières durent une heure et demie et la troisième ne dure qu'une heure.\par
Les qualifications se divisent en trois parties :
\begin{itemize}
\item  la première partie dure 18 minutes et, à la fin, les 5 dernières voitures se retrouvent éliminées des qualifications;
\item la deuxième partie dure 15 minutes et, à la fin, les 5 dernières voitures se retrouvent aussi éliminées;
\item la troisième et dernière partie dure 12 minutes.
\end{itemize}

Le classement des qualifications est calculé sur base des meilleurs temps de parcours du circuit ce qui permet le placement des voitures sur la grille de départ de la course.\par
La course détermine le gagnant. Contrairement aux périodes d'essais et aux qualifications, c'est un nombre de tours qui définit la durée de la course. Ce nombre est défini en divisant la distance réglementaire de 305 kilomètres par la taille du circuit. Durant cette course, les voitures ont l'obligation de s'arrêter au stand au moins une fois. Ce stand se situe dans le troisième secteur du circuit.\\
\\

\newpage
\part{Analyse}

\section{Plan de l'application}

L'architecture de notre programme se définit comme suit :
\begin{itemize}

\item RaceWE : c'est le fichier principal, il gère la création des segments de mémoire partagée, des sémaphores, des processus,... On y gère aussi l'affichage et le déroulement du week-end de course.
\item Practice : ce fichier contient les fonctions nécessaires au déroulement des essais.
\item Qualification : ce fichier contient les fonctions nécessaires au déroulement des qualifications.
\item Race : ce fichier contient les fonctions nécessaires au déroulement de la course. 
\item Utilitaire : ce fichier contient toutes les fonctions utilitaires qui sont utilisées à plusieurs endroits dans le programme. C'est aussi ici que sont déclarées les variables globales.
\item Voiture : ce fichier contient la structure utilisée pour représenter les voitures, ainsi qu' une fonction d'initialisation et de réinitialisation des voitures.

\end{itemize}

\section{Création du random}
Le random est la fonction la plus utilisée dans le projet. D'une part, nous l'utilisons pour générer les vitesses des voitures en fonction d'un maximum et d'un minimum dépendant de la section du circuit. D'autre part, il permet la gestion du crash (0,1\%) et de l'arrêt au stand (15\%).
Pour l'utiliser, il nous faut planter une graine sinon les valeurs générées par la fonction seront tirées d'une même suite de valeur. Voici le code pour planter la graine:
\begin{lstlisting}[language=c]
srand(time(NULL)+getpid());
\end{lstlisting}
On plante cette graine au début des processus enfants. La dépendance au pid permet d'avoir une fonction random différente pour chaque voiture.\par

La fonction permettant la génération d'un nombre aléatoire contient un modulo permettant de définir l'intervalle de valeur en fonction des paramètres. Voici le code pour cette génération d'aléas:
\lstinputlisting[language=c, firstline=40, lastline=43]{utilitaire.c} 

\section{Gestion de la mémoire}
Étant donné que chaque voiture est un processus, la mémoire partagée permet de gérer une zone mémoire commune à tous ces processus et d'en garder les valeurs de sorties. Nous utilisons 5 segments de mémoires partagés différents :
\begin{enumerate}

\item un tableau de voitures;
\item un tableau reprenant les variables globales qui doivent pouvoir être modifiées par tous les processus;
\item un tableau des pids, qui nous permet de faire correspondre un processus à une voiture;
\item un tableau de voiture qui peuvent prendre part au départ de la deuxième qualification;
\item un tableau de voiture qui peuvent prendre part au départ de la troisième qualification.

\end{enumerate}
\par
Pour synchroniser l'accès à la mémoire partagée, nous avons utilisé les sémaphores avec l'algorithme de Lamport alias l'algorithme de la boulangerie. Quand un processus entre dans la section critique, le sémaphore passe à 1 s'ensuit la mise en attente de tous les autres processus. Après la libération de la zone critique, le sémaphore passe à zéro, pour permettre aux autres processus de pouvoir entrer dans la zone critique.
\begin{lstlisting}[language=c]
	id_sem = semop(id_sem, &semWait, 1);
	id_sem = semop(id_sem, &semDo, 1);
	//Ce qui doit ce jouer 
	id_sem = semop(id_sem, &semPost, 1);
\end{lstlisting}
\newpage
\part{Conclusion}
\section{Difficultés rencontrées}

La première difficulté rencontrée a été de comprendre au mieux les consignes afin de diviser le programme en modules. Nous nous sommes donc réunis pour diviser le problème et pour nous accorder sur les différentes spécificités de ces différentes parties.\par
Ensuite, il a fallu trouver une méthode pour générer un nombre aléatoire, élément essentiel de ce projet. Nous avons tout de suite constaté que ce n'était pas possible de faire du "vrai" aléatoire en language C. Nous avons rapidement constaté qu'il fallait planter une graine (seed) afin de donner une impression de nombre tiré au hasard.\par 
Par la suite, un bug nous a donné du fil à retordre :  quand on sortait de la boucle des essais, toutes les valeurs des voitures se remettaient à 0. Il s'agissait en réalité d'une mauvaise utilisation des pointeurs en combinaison avec la mémoire partagée. En effet, les différents processus ont tous des plages de mémoires différentes. Quand on leur donne un pointeur, ils changent les valeurs contenues à l'adresse indiquée par ce pointeur. Le problème est que cette adresse est une adresse dans la plage d'adresse locale du processus, la valeur n'est donc pas vraiment modifiée dans la mémoire partagée. En conséquence, quand on revient au processus père, les valeurs sont toujours à 0 car elles n'ont en fait pas été modifiées.

\section{Évolution future de l'application}

Une amélioration possible serait d'implémenter une interface graphique comprenant une animation de la course pour mieux visualiser les déplacements, les dépassements, les arrêts au stand ainsi que le tableau des scores. De plus, cette interface pourrait être synchronisée pour avoir un rendu plus proche de la réalité.

\section{Conclusion}
Ce projet nous a permis de mettre en pratique la mémoire partagée ainsi que les sémaphores comme vu lors du cours théorique. Nous avons réussi à implémenter l'écriture dans un fichier afin de garder une trace de la course, la concurrence entre processus ainsi que la gestion des sémaphores et de la mémoire partagée.

\newpage
\appendix
\part{Code}
\section{raceWE.c}
\lstinputlisting[language=C]{raceWE.c}

\section{voiture.h}
\lstinputlisting[language=C]{voiture.h}
\section{voiture.c}
\lstinputlisting[language=C]{voiture.c}

\section{utilitaire.h}
\lstinputlisting[language=C]{utilitaire.h}
\section{utilitaire.c}
\lstinputlisting[language=C]{utilitaire.c}

\section{practice.h}
\lstinputlisting[language=C]{practice.h}
\section{practice.c}
\lstinputlisting[language=C]{practice.c}

\section{qualification.h}
\lstinputlisting[language=C]{qualification.h}
\section{qualification.c}
\lstinputlisting[language=C]{qualification.c}

\section{race.h}
\lstinputlisting[language=C]{race.h}
\section{race.c}
\lstinputlisting[language=C]{race.c}

\end{document}
